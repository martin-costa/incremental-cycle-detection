\documentclass{article}

% packages
\usepackage[utf8]{inputenc}
\usepackage{amsthm}
\usepackage{amsmath}
\usepackage{amssymb}
\usepackage{enumerate}

\usepackage{calrsfs}
\DeclareMathAlphabet{\pazocal}{OMS}{zplm}{m}{n}

% definitions and commands
\newtheorem{theorem}{Theorem}
\newtheorem{lemma}{Lemma}
\newtheorem{corollary}{Corollary}
\newtheorem{proposition}{Proposition}
\newtheorem{definition}{Definition}

\newcommand{\norm}[1]{\left\lVert#1\right\rVert}

\voffset = -20pt % defaults to 0pt
\textheight = 581pt % defaults to 561pt
\hoffset = -10pt % defaults to 0pt
\textwidth = 375pt % defaults to 355pt

\title{Upper Bounds on Expected Recourse of Incremental Cycle Detection Algorithms (Draft 1)}
\author{Martin Costa}
\date{August 2020}

\begin{document}

\maketitle

% CHAPTER 1
\section{Introduction}

\subsection{Basic Definitions}

Let $G=(V,E)$ be a directed acyclic graph (DAG). Given some countable set $X$, denote by $\mathcal{S}_{X}$ the space of all sequences over $X$ where each element of $X$ appears exactly once. We call the elements of $\mathcal S_E$ the \textit{insertion sequences} of $G$ and the elements of $\mathcal S_V$ the \textit{orderings} of $G$. Notice that there is a clear one-to-one correspondence between total orderings of $V$ and elements of $\mathcal S_V$.

Denote by $G^{\pi}_{t}$ the graph $(V, \{ e_{\pi(1)},...,e_{\pi{(t})} \})$, i.e the graph after the first $t$ insertions from $E_{\pi}$.

Let $G=(V,E)$ be a directed acyclic graph (DAG) with $V=\{ u_{1},...,u_{n}\}$ and $E=\{ e_{1},...,e_{m}\}$. Denote by $\mathcal{S}_{E}$ the space of all sequences over $E$ where each element of $E$ appears exactly once. For each $\pi \in \pazocal{S}_{m}$ (the set of all bijections from $[m]$ to itself), we associate a sequence $(e_{\pi(1)},...,e_{\pi{(m})})\in \mathcal{S}_{E}$, denoted by $E_{\pi}$. This correspondence is bijective so given any sequence we also implicitly associate a permutation. Denote by $G^{\pi}_{t}$ the graph $(V, \{ e_{\pi(1)},...,e_{\pi{(t})} \})$, i.e the graph after the first $t$ insertions from $E_{\pi}$.

For each $\pi \in \pazocal{S}_{m}$, let $recourse_{G}(\pi) \in \mathbb{N}^{m}$ denote the \textit{recourse vector of $\pi$}, such that for $t \in [m]$, $recourse_{G}(\pi)_{t}$ denotes the recourse from the insertion of $e_{\pi(t)}$ during the insertion of the sequence of edges $E_{\pi}$.

Define the \textit{recourse} of some sequence in $\mathcal{S}_{E}$ by $R_{G}(\pi) \triangleq \norm{recourse_{G}(\pi)}_{l^1}$, i.e. as the sum of the recourse of all the individual insertions. Define $W(G) \triangleq \sup_{\pi \in \pazocal{S}_{m}} {R_{G}(\pi)}$, i.e as the largest recourse that can be attained by any sequence of edge insertions of $E$. For any graph $H$, define $\Psi(H) \triangleq \sum_{u \in V(H)}\vert reach_{H}^{-1}(u) \cup reach_{H}(u) \vert \leq n^{2}$.

\subsection{Trivial Recourse Bound}

Fix some DAG $G=(V,E)$, using the same notation as above for $V$ and $E$, and any $\pi \in \pazocal{S}_{m}$. Suppose that during the run of the simple greedy (2-way search) algorithm on input $E_{\pi}$, the edge $(u,v)$ is inserted with recourse $\lambda$. Let $D$ denote the descendants of $v$ discovered by the forward search done by the algorithm during the insertion, and $A$ denote the ancestors of $u$ discovered by the backwards search done by the algorithm during the insertion. Before the insertion, at most one edge from $D$ could reach at most one edge in $A$, but after the insertion, all of the nodes in $D$ can reach all of the nodes in $A$. So the descendants of $\Omega(\vert A \vert)$ nodes increase by $\Omega(\vert D\vert)$ and the ancestors of $\Omega(\vert D\vert)$ nodes increase by $\Omega(\vert A \vert$). Combined with the fact that $D = \Theta(A)$ and $\lambda \leq \vert A\vert + \vert D\vert$ we get that for all $t \in [m]$, $\Psi(G_{t-1}^{\pi}) + recourse_{G}(\pi)_{t}^{2} = \pazocal{O}(\Psi(G_{t}^{\pi}))$.

Repeatedly applying the inequality above (and noticing that $\Psi(G_{0}^{\pi})=0$), we can deduce that $\norm{recourse_{G}(\pi)}_{l^2}^{2} = \sum_{t=1}^{m}{recourse_{G}(\pi)^{2}_{t}} = \pazocal{O}(\Psi(G_{t}^{\pi}))$. Using some simple norm bounds, it follows that $\norm{recourse_{G}(\pi)}_{l^{1}} \leq \sqrt{m}\norm{recourse_{G}(\pi)}_{l^{2}} = \pazocal{O}(\sqrt{m\Psi(G_{t}^{\pi})}) = \pazocal{O}(n\sqrt{m})$. So for any graph $G$, we have $W(G)=\pazocal{O}(n\sqrt{m})$.

% CHAPTER 1
\section{Expected Recourse}

Throughout this section we implicitly assume we are running $\pazocal A_2$.

\subsection{Worst Case Recourse}

In the work of [HKMST11], they construct a DAG $G=(V,E)$ and an insertion sequence $\pazocal E \in \mathcal{S}_{E}$ such that $rec(\pazocal E) = \Omega(n\sqrt{m})$, proving that the simple greedy 2-way search algorithm $\pazocal A_2$ does not satisfy worst case recourse $\pazocal{\tilde{O}}(n)$ for all graphs. However, this example does have an \textit{expected recourse} of $\pazocal{\tilde{O}}(n)$, where the expected recourse is

\[ \mathbb{E}_{\pazocal E \in \mathcal{S}_{E}}[rec(\pazocal E)] = \sum_{j \geq 0} j \cdot \mathbb{P}[rec=j] = \frac{1}{m!} \sum_{\pazocal E \in \mathcal{S}_{E}}rec(\pazocal E) \]

We will now prove that this is the case.

\subsection{Expected Recourse Example}

Consider the graph $G=(V,E)$ defined as follows. Let $k \in \mathbb{N}$ and let $S_{i} = \{ x_{i}, z_{i}^{2}, ..., z_{i}^{k-1}, y_{i} \}$ be sets of nodes of size $k$ such that $V = \bigcup_{i=1}^{k}S_{i}$. Starting with $E= \varnothing$, for all $i \in [k]$ add the edges $(x_{i}, z_{i}^{2}), (z_{i}^{2}, z_{i}^{3}), ..., (z_{i}^{k-2}, z_{i}^{k-1}), (z_{i}^{k-1}, y_{i})$ to $E$. Let $X_{i}=\{x_{i},...,x_{k}\}$ and $Y_{i}=\{y_{1},...,y_{i}\}$. For all $i \in [k], j \in X_{i+1}$ add $(y_{i}, x_{j})$ to $E$ and for all $i \in [k], j \in Y_{i-1}$ add $(y_{j}, x_{i})$ to $E$. We have $n=k^2$ and $m=\frac{3}{2}k(k-1)$. We refer to the $y_{i}$ as \textit{heads} and to the $x_{i}$ as \textit{tails}.

Suppose we run $\pazocal A_2$ on this graph with some insertion sequence $\pazocal E \in \mathcal S_E$. We want to classify different types of edge insertions and show that we expect $\pazocal{\tilde{O}}(n)$ total recourse from each type. Suppose we insert the edge $e_{\pi(t)}=(u,v)$ into the graph $G_{t-1}^{\pi}$, then we classify this edge insertion as follows.

\begin{enumerate}[Type 1.]
    \item If for some $i \in [k]$, $u,v \in S_{i}$ and $G_{t}^{\pi}[S_{i}]$ is \textbf{not} a weakly connected graph
    \item If for some $i \in [k]$, $u,v \in S_{i}$ and $G_{t}^{\pi}[S_{i}]$ is a weakly connected graph
    \item If $u$ is a head and $v$ is a tail
\end{enumerate}

It should be clear that any edge insertion is exactly one of these 3 types. Denote by $R_{G}^{j}(\pi)$ the total recourse of all type $j$ insertions from the sequence of edges $E_{\pi}$. So we have that $R_{G}(\pi) = \sum_{j=1}^{3}R_{G}^{j}(\pi)$ and hence $\mathbb{E}[R_{G}] = \sum_{j=1}^{3}\mathbb{E}[R_{G}^{j}]$.

The outline for the proof that the expected runtime of the simple greedy algorithm on this input is $\pazocal{\tilde{O}}(n)$ is as follows.

\begin{enumerate}[I.]
    \item Show that $R_{G}^{1}(\pi)=\pazocal{\tilde{O}}(n)$ giving us that $\mathbb{E}[R_{G}^{j}]=\pazocal{\tilde{O}}(n)$.
    \item Show that the expected number of type 2 insertions in the first $m-k(\log n)^2$ insertions is $k^{1-\frac{8}{3}\log k}$.
    \item Show that the expected recourse of a type 3 insertion before any type 2 insertions have occurred is $\pazocal{\tilde{O}}(n)$.
    \item Using the fact that type 3 insertions occur uniformly at random, bound the recourse for the last $k(\log n)^2$ insertions by $\pazocal{\tilde{O}}(n) $.
    \item Combine everything to deduce that $\mathbb{E}[R_{G}]=\pazocal{\tilde{O}}(n)$.
\end{enumerate}

\subsubsection{Step I}

We first start off by breaking down type 1 insertions into 2 further types. Suppose we insert an edge like above, and it is classified as a type 1 insertion, then by definition of a type 1 insertion, we can assume that for some $i \in [k]$, $u,v \in S_{i}$ and $G_{t}^{\pi}[S_{i}]$ is not a weakly connected graph. We further classify this edge insertion as follows.

\begin{enumerate}[Type {1}.1]
    \item one of $x_{i}$ or $y_{i}$ is contained in $reach^{-1}_{G_{t}^{\pi}}(u) \bigcup reach_{G_{t}^{\pi}}(v)$
    \item neither of $x_{i}$ or $y_{i}$ is contained in $reach^{-1}_{G_{t}^{\pi}}(u) \bigcup reach_{G_{t}^{\pi}}(v)$
\end{enumerate}

It should again be clear that all type 1 insertions are exactly one of these 2 types. The weakly connected components of the graph $G_{t}^{\pi}[S_{i}]$ are simple paths, so we can think of type 1 insertions as appending these paths together. Furthermore, we can think of type 1.1 insertions as appending paths that don't contain either $x_{i}$ or $y_{i}$, and type 1.2 insertions as appending a path that doesn't contain either $x_{i}$ or $y_{i}$ with one that does. 

Since the only nodes in $S_{i}$ that may be weakly connected to nodes not contained in $S_{i}$ are $x_{i}$ and $y_{i}$, we can deduce that the recourse of a type 1.1 insertion is at most the length of the smallest path being appended (with the length being how many nodes it contains). Similarly, we can deduce that the recourse of a type 1.2 insertion is at most the length of the path being appended that doesn't contain either $x_{i}$ or $y_{i}$.

Hence, within some set $S_{i}$, we get that the total recourse of the type 1.1 insertions is $\pazocal{O}(k \log k)$, and the total recourse of the type 1.2 insertions is $\pazocal{O}(k)$. Adding up over all the $S_{i}$ for all $i \in [k]$, we get that the total recourse of the type 1 insertions is $\pazocal{\tilde{O}}(n)$, i.e $R_{G}^{1}(\pi)=\pazocal{\tilde{O}}(n)$. So we get $\mathbb{E}[R_{G}^{j}]=\pazocal{\tilde{O}}(n)$.

\subsubsection{Step II}

Given any $S_{i}$, what is the expected value of $\vert S_{i} \cap reach_{G_{t}^{\pi}}(x_{i})\vert$? Equivalently, how many nodes in $S_{i}$ do we expect $x_{i}$ to be able to reach after $t$ insertions? It should be clear that there is exactly one type 3 insertion per each set $S_{i}$, and that there has been a type 3 insertion in the set $S_{i}$ if and only if $\vert S_{i} \cap reach_{G_{t}^{\pi}}(x_{i})\vert = k$. Hence we get that the probability that there has been as type 3 insertion in the set $S_{i}$ is equal to the probability that $\vert S_{i} \cap reach_{G_{t}^{\pi}}(x_{i})\vert = k$. We denote the probability that $\vert S_{i} \cap reach_{G_{t}^{\pi}}(x_{i})\vert = j$ for some $j \in \{0,...,k\}$ by $p_{j}(t)$ (note that the probability does not depend on the value of $i$, i.e. which set it is). We can show that

\begin{align*}
p_{0}(t) = 0,     \quad 1 \leq j \leq k-1 \quad p_{j}(t) = \binom{m-j}{t-j+1}\binom{m}{t}^{-1}, \quad p_{k}(t) = \binom{m-k+1}{t-k+1}\binom{m}{t}^{-1}
\end{align*}

From now on we fix the value $\gamma=m-k(\log n)^2$. We now find the expected value of $k \cdot p_{k}(\gamma)$.

\[k \cdot p_{k}(\gamma) = k\binom{m-k+1}{\gamma-k+1}\binom{m}{\gamma}^{-1} = k\prod_{j=\gamma + 1}^{m} \bigg( \frac{j-k+1}{j} \bigg) \] 

\[ \leq k \bigg(\frac{m-k+1}{m}\bigg)^{m-\gamma} = k\bigg(1 - \frac{2}{3k}\bigg)^{4k (\log k)^2} \leq ke^{-\frac{8}{3}(\log k)^2} = k^{1-\frac{8}{3}\log k}\] 

\subsubsection{Step III}

We can use the fact that we do not expect any type 2 insertions to occur within the first $\gamma$ insertions to help us bound the expected recourse of the type 3 insertions in this region. An edge $(u,v)$ causes a type 3 insertion if and only if $u$ is a head and $v$ is a tail. The recourse of such an insertion is at most the smallest of $\vert reach_{G_{t}^{\pi}}(v)\vert$ and $\vert reach^{-1}_{G_{t}^{\pi}}(u)\vert$. We expect this value to be less than $k$ since we do not expect any type 2 insertions to have occurred, and hence any tail $x_{i}$ to be able to reach its corresponding head $y_{i}$, and hence any node not in $S_{i}$. But how many nodes exactly do we expect it to be able to reach? This will be expected value of $\vert S_{i} \cap reach_{G_{t}^{\pi}}(x_{i})\vert$ which is equal to $\sum_{i=0}^{k}k \cdot p_{k}$. We now show that $\sum_{t=0}^{m} \sum_{i=0}^{k}k \cdot p_{k}(t) = \pazocal{O}(n \log n)$, hence we expect all the type 3 insertions in this region to have a total recourse of at most $\pazocal{\tilde{O}}(n)$, since we expect their total recourse to be upper bounded by $\pazocal{O}(n \log n)$.

\[\sum_{i=0}^{k}k \cdot p_{k}(t) = \Bigg(k\binom{m-k+1}{t-k+1} + \sum_{i=0}^{k-1}i\binom{m-i}{t-i+1} \Bigg)\binom{m}{t}^{-1} \]

\[ = k\prod_{j=t+1}^{m}\bigg(\frac{j-k+1}{j}\bigg) + (m-t)\sum_{i=1}^{k-1}\frac{i}{m-i+1} \prod_{j=t+1}^{m}\bigg(\frac{j-i+1}{j}\bigg) \]

\[ = k\prod_{j=1}^{k-1}\bigg(\frac{t-j+1}{m-j+1}\bigg) + (m-t)\sum_{i=1}^{k-1}\frac{i}{m-i+1} \prod_{j=1}^{i-1}\bigg(\frac{t-j+1}{m-j+1}\bigg) \]

\[ \leq k\Big( \frac{t}{m} \Big)^{k-1} + \frac{m-t}{m-k+1}\sum_{i=1}^{k-1}i\Big( \frac{t}{m} \Big)^{i-1} \]

\[ \leq k\bigg(1 - \frac{m}{m-k+1} \bigg)\Big( \frac{t}{m} \Big)^{k-1} + \frac{m^2}{(m-t)(m-k+1)} \bigg(1-\Big( \frac{t}{m} \Big)^{k} \bigg) \]

\[ \leq \frac{m^2}{(m-t)(m-k+1)}\]

Since the rational function on the last line is strictly increasing when considered as a function of $t$, we get that

\[ \sum_{t=0}^{m} \sum_{i=0}^{k}k \cdot p_{k}(t) \leq k + \int_{0}^{m-1} \frac{m^2}{(m-t)(m-k+1)} dt = k + \frac{m^2 \log m}{m-k+1} = \pazocal{O}(n \log n)\]



\subsubsection{Step IV}

For any subgraph of $H$ of $G$ with $V(H)=V(G)$, define $\Psi^{*}(H) \triangleq \sum_{i=1}^{k} \vert (Y_{i-1} \cap reach^{-1}_{H}(x_{i})) \cup (X_{i+1} \cap reach_{H}(y_{i})) \vert$. Notice that $\Psi^{*}(G_{t}^{\pi})$ is at least 2 times the number of type 3 insertions that have occurred and that $\Psi^{*}(G) = k(k-1)$.

Combining the fact that the $\frac{1}{2}k(k-1)$ edges that cause type 3 insertions are all pre-determined (iff they are from a head to a tail), with the fact that edges are inserted uniformly at random, we get that we expect $\frac{1}{3}\gamma = \frac{1}{2}k(k-1) - \frac{1}{3}k(\log n)^2$ of the first $\gamma$ insertions to be of type 3. So we get that we expect $\Psi^{*}(G)-\Psi^{*}(G_{\gamma}^{\pi}) = \pazocal{O}(k(\log n)^2)$. We show that if this is that case, then the recourse of the last $m-\gamma$ insertions is $\pazocal{\tilde{O}}(n)$.

Consider any edge $(u,v)$ of the last $m - \gamma$ edges inserted into the graph, whose insertion has recourse $\lambda$. Let $F,B \subseteq \{S_{1},...,S_{k}\}$ be the collections of the sets discovered by the algorithm during the forward and backwards searches respectively after the insertion. Since each of the searches discovers $\Omega(\lambda)$ nodes and each $S_{i}$ contains $k$ nodes, we know that $\vert F \vert, \vert B \vert = \Omega(\lambda/k)$. Note that $F$ and $B$ are disjoint, since if not, for some $i$, $S_{i} \in F \cap B$, so eventually we will have $y_{i} \rightsquigarrow u \rightsquigarrow v \rightsquigarrow x_{i} \rightsquigarrow y_{i}$ giving us a cycle. Also note that none of the heads in sets from $B$ can reach any of the tails in the sets from $F$ before this insertion, but after the insertion all heads in the sets from $B$ can reach all the tails in the sets from $F$. This gives us that if $e_{\pi(t)}=(u,v)$ then $\Psi^{*}(G_{t-1}^{\pi}) + 2\vert F \vert\vert B \vert \leq \Psi^{*}(G_{t}^{\pi})$, hence $(\lambda/k)^2 = \pazocal{O}(\vert F \vert\vert B \vert) = \pazocal{O}(\Psi^{*}(G_{t}^{\pi}) - \Psi^{*}(G_{t-1}^{\pi}))$.

Let $\lambda_{\gamma},...,\lambda_{m}$ be the recourse of the last $m - \gamma$ insertions. Applying the equation above we get that

\[\sum_{j=\gamma}^{m}\bigg(\frac{\lambda_{j}}{k}\bigg)^{2} = \sum_{j=\gamma}^{m} \pazocal{O}\big(\Psi^{*}(G_{j}^{\pi}) - \Psi^{*}(G_{j-1}^{\pi})\big) 
= \pazocal{O}\big(\Psi^{*}(G) - \Psi^{*}(G_{\gamma}^{\pi})\big) = \pazocal{O}\big(k (\log n)^2\big)\]

Applying some norm bounds we get

\[\sum_{j=\gamma}^{m}\lambda_{j} \leq \bigg((m-\gamma+1)\sum_{j=\gamma}^{m}\lambda_{j}^{2}\bigg)^{\frac{1}{2}} = \sqrt{\pazocal{O}(k(\log n)^2)}\pazocal{O}\big(\sqrt{k^{3}(\log n)^2}\big) =\pazocal{O}(n(\log n)^2)\]

Hence the total recourse of the last $k(\log n)^2$ insertions is expected to be $\pazocal{\tilde{O}}(n)$.

\subsubsection{Step V}

Combining these results, we expect the total recourse from the first $m-k(\log n)^2$ insertions to be $\pazocal{\tilde{O}}(n)$ since: type 1 insertions have a total recourse of $\pazocal{\tilde{O}}(n)$, with high probability there will be no type 2 insertions in this region, and the type 3 insertions in this region have expected total recourse $\pazocal{\tilde{O}}(n)$. We also shown that we expect the total recourse of the last $k (\log n)^2$ insertions to be $\pazocal{\tilde{O}}(n)$. It follows that the expected total recourse is equal to $\pazocal{\tilde{O}}(n)$.

\end{document}