\documentclass{article}

% packages
\usepackage[utf8]{inputenc}
\usepackage{amsthm}
\usepackage{amsmath}
\usepackage{amssymb}
\usepackage{enumerate}
\usepackage{dsfont}

\usepackage{calrsfs}
\DeclareMathAlphabet{\pazocal}{OMS}{zplm}{m}{n}

% definitions and commands
\newtheorem{theorem}{Theorem}
\newtheorem{lemma}{Lemma}
\newtheorem{corollary}{Corollary}
\newtheorem{proposition}{Proposition}
\newtheorem{definition}{Definition}

\newcommand{\norm}[1]{\left\lVert#1\right\rVert}

\voffset = -20pt % defaults to 0pt
\textheight = 581pt % defaults to 561pt
\hoffset = -10pt % defaults to 0pt
\textwidth = 375pt % defaults to 355pt

\title{Expected Recourse Bounds (Draft 1)}
\author{Martin Costa}
\date{August 2020}

\begin{document}

\maketitle

% CHAPTER 1
\section{Lemma Statements}

We want to prove the following Lemmas. Fix DAG $G=(V,E)$ with root $r$.

\begin{lemma}
Let $(e_t)_{t \in [m]}$ be any normal insertion sequence of $G$. The expected recourse of $r$ is $\pazocal{O}(\log n)$, with expectation taken over the uniformly generated initial ordering.
\end{lemma}

\begin{lemma}
Let $(e_t)_{t \in [m]}$ be any insertion sequence of $G$ and fix any initial ordering $\prec_0$. We can compute a normal insertion sequence $\Gamma((e_t)_{t \in [m]}) = (\hat{e}_t)_{t \in [m]}$ such that for all $x \in V$ we have $recourse_x((e_t)_{t \in [m]}, \prec_0) \leq recourse_x((\hat{e}_t)_{t \in [m]}, \prec_0)$.
\end{lemma}

% CHAPTER 2
\section{Lemma 2}

We start by giving some definitions. Fix DAG $G=(V,E)$ with root $r$.

\begin{definition}
Given some insertion sequence $(e_t)_{t \in [m]}$ of $G$ the \textbf{activation sequence} of $(e_t)_t$ with respect to $r$, denoted by $(a_t)_t = act_r((e_t)_t)_t$, is a sequence of length $m$ of (potentially empty) sets that form a partition of $V$, where $(a_t)_t = reach_{t}^{-1}(r) \setminus reach_{t-1}^{-1}(r)$.
\end{definition}

Given some activation sequence $(a_t)_t$, let $a(x)$ denote the set in the sequence that contains $x$ and let $a^{-1}_x$ denote the value of $t$ such that $x \in a_t$. Notice that the activation sequence of some insertion sequence does not depend on the underlying ordering.

\begin{definition}
Given two activation sequences $(a_t)_t = act_r((e_t)_t)_t$ and $(\hat{a}_t)_t = act_r((\hat{e}_t)_t)_t$ we say that $(a_t)_t$ is a \textbf{compression} of $(\hat{a}_t)_t$ (or that $(\hat{a}_t)_t$ is an \textbf{expansion} of $(a_t)_t$) if for all $x,y \in V$, if $a^{-1}_x < a^{-1}_y$ then $\hat{a}^{-1}_x < \hat{a}^{-1}_y$. We denote this by $(a_t)_t \trianglelefteq (\hat{a}_t)_t$.
\end{definition}

We can see that compression of activation sequences gives a partial order over the space of activation sequences for some graph.

\begin{proposition}
Given insertion sequences $(e_t)_t$ and $(\hat{e}_t)_t$, and corresponding activation sequences $(a_t)_t = act_r((e_t)_t)_t$, $(\hat{a}_t)_t = act_r((\hat{e}_t)_t)_t$ then for any initial ordering $\prec_0$ of $V$
\[ (a_t)_t \trianglelefteq (\hat{a}_t)_t  \Rightarrow recourse_r((e_t)_t,\prec_0) \leq recourse_r((\hat{e}_t)_t, \prec_0) \]
\end{proposition}

\begin{proposition}
Given any insertion sequence $(e_t)_t$
\[ act_r((e_t)_t)_t \trianglelefteq act_r(\Gamma((e_t)_t))_t \]
\end{proposition}

\end{document}

\begin{lemma}
There is a surjection from the space of normal insertion sequences of $G$ to the space of fully decompressed activation sequences of $G$ which is bijective if and only if $G$ is a tree.
\end{lemma}