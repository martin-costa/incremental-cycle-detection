\documentclass{article}

% packages
\usepackage[utf8]{inputenc}
\usepackage{amsthm}
\usepackage{amsmath}
\usepackage{amssymb}
\usepackage{enumerate}
\usepackage{dsfont}

\usepackage{calrsfs}
\DeclareMathAlphabet{\pazocal}{OMS}{zplm}{m}{n}

% definitions and commands
\newtheorem{theorem}{Theorem}
\newtheorem{lemma}{Lemma}
\newtheorem{corollary}{Corollary}
\newtheorem{proposition}{Proposition}
\newtheorem{definition}{Definition}

\newcommand{\norm}[1]{\left\lVert#1\right\rVert}

\voffset = -20pt % defaults to 0pt
\textheight = 581pt % defaults to 561pt
\hoffset = -10pt % defaults to 0pt
\textwidth = 375pt % defaults to 355pt

\title{Expected Recourse Bounds (Draft 1)}
\author{Martin Costa}
\date{August 2020}

\begin{document}

\maketitle

% CHAPTER 1
\section{Introduction}

The main idea of these definitions and propositions will be to give us tools which we can use to analyse the expected recourse of the \textit{simple consistent one-way search} algorithm, which we will denote by $\pazocal{A}$. We assume from now on that $\pazocal{A}$ always performs forward searches, and we fix a DAG $G=(V,E)$.

\begin{definition}
During the run of $\pazocal{A}$ on $G$, we say a node $u$ is \textbf{critical} to $v$ if:
\begin{enumerate}
    \item $u$ cannot reach $v$
    \item there exists some edge $(u,w) \in E$ and $w$ can reach $v$ 
\end{enumerate}
We say an edge $(u,v)$ is \textbf{critical} to $w$ if its insertion increases the amount of nodes that can reach $w$, this implies that $u$ is critical to $w$. Similarly, we say that an edge $(u,v)$ is \textbf{right critical} to $w$ if its insertion causes the recourse of $w$ to increase.
\end{definition}

This definition has the following obvious properties.

\begin{proposition}
Given any $u,v \in V$, if $u$ is \textbf{not} an ancestor of $v$ in $G$, then $u$ can never be critical to $v$ during the run of $\pazocal{A}$ on $G$. 
\end{proposition}

\begin{proposition}
During the run of $\pazocal{A}$ on $G$, the insertion of an edge $(u,v)$ can only increase the recourse of some node $w$ if $(u,v)$ is critical to $w$.
\end{proposition}

Combining these two properties we get the following.

\begin{lemma}
Fix a sequence of edge insertions and an initial ordering for $G$, and use these to induce a sequence of edge insertions and an initial ordering for $H = G[reach^{-1}_{G}(x)]$ for some node $x$. Then given some $u \in reach^{-1}_{G}(x)$, the recourse of $u$ in the run of $\pazocal{A}$ on $G$ equals the recourse of $u$ in the run of $\pazocal{A}$ on $H$.
\end{lemma}

Intuitively, this lemma is saying that the effect of $\pazocal{A}$ on some node $u$ depends only on the structure of the ancestors of $u$. We can immediately notice the following.

\begin{corollary}
In the run of $\pazocal{A}$ on $G$, there always exists some node $u$ whose removal has no effect on the recourse of any other node in the graph.
\end{corollary}

Combining the Lemma and the Corollary it may be possible to approach this analysis inductively, considering only the predecessors of some node $u$ and noticing that their behaviour can not be influenced by the presence of $u$, and then considering the effect that they will have upon $u$. Motivated by this Lemma, I also believe it is the case that the recourse of $u$ is $\pazocal{O}(\log{\vert reach_{G}^{-1}(u) \vert})$, since the removal of any node not in this set from $G$ will not effect the recourse of $u$.

% CHAPTER 2
\section{Memorylessness}

\begin{definition}
Let a \textbf{configuration} of the run of $\pazocal{A}$ on $G$ be a tuple $(S,\prec)$ where $S$ is a subset of $E$ and $\prec$ is a topological ordering of $V$ with respect to $G[S]$. We say a configuration $(S,\prec)$ \textbf{yields} $(S^*,\prec^*)$ if there exists some sequence of edge insertions of $S^* \setminus S$ that can be inserted by $\pazocal{A}$ to obtains order $\prec^*$ starting from $\prec$.
\end{definition}

\begin{definition}
We say an algorithm is \textbf{memoryless} if given any configuration $(S, \prec)$ which yields configurations $(S^*, \prec^*_1)$ and $(S^*, \prec^*_2)$, we have that $\prec^*_1 = \prec^*_2$.
\end{definition}

\begin{proposition}
$\pazocal{A}$ is memoryless on the space of forests of rooted trees.
\end{proposition}

\begin{proof}
We proceed by induction on the depth $D$ of the deepest tree in the forest. The base case $D=0$ is trivial as there are no insertions to be made.

Let $\pazocal{F}$ be a forest of rooted trees with maximum depth $k$ and assume that the proposition holds for all $0 \leq D < k$. Let $X$ be the set of nodes in $\pazocal{F}$ of out degree 0. By Lemma 1, we know that the presence of the $x_i$ has no effect on how $\pazocal{A}$ behaves with the rest of the nodes $Y = V \setminus X$ in $\pazocal{F}$. So, by applying the inductive hypothesis, we get that the positions of any $u,v \in Y$ relative to each other in the ordering is unaffected by the order in which all edges without an endpoint in $X$ are inserted.

Now consider some fixed $x \in X$ with children $y_1,...,y_l$ in $\pazocal{F}$ and consider the following 2 cases.

\begin{enumerate}
    \item $x$ is to the right of all of its ancestors in the initial ordering
    \item $x$ is to the left of one of its ancestors in the initial ordering
\end{enumerate}

In case 1, $x$ is never moved by $\pazocal{A}$, so no insertions with an endpoint in $x$ have any effect on the ordering. In case 2, the position of $x$ in the final ordering is just to the right of the rightmost node out of $y_1,...,y_l$. In either case, the position of $x$ depends only on the edges present in $\pazocal{F}$, never on when they are inserted.
\end{proof}

Using the memoryless property of rooted trees we can prove the following lemma which will be crucial for the next section. First we define a potential function. Fix a rooted tree $\pazocal{T} = (V,E)$ with root $r$, and let $A \subseteq V$ be a set of mutually unrelated nodes in $\pazocal T$, i.e. given any two nodes from $A$, there exists no path between them in $\pazocal T$. Let $\Psi_A \triangleq \vert Child(A) \cap R(A) \vert$, where $Child(A)$ is the set of children of $A$ and $R(A)$ is the set of all nodes to the right of $A$ in the topological ordering. Let $\Psi_A(t)$ denote the value of $\Psi_A$ after $t$ edges with endpoints in $A$ have been inserted.

We show that the memoryless property of rooted trees implies the following probabilistic recurrence relation between the expected values of the potential function. Let $z^+$ denote $max(z, 0)$.

\begin{proposition}
$\mathbb{E}[\Psi_A(t+1)] = \frac{1}{2}(\mathbb{E}[\Psi_A(t)] - 1)^+$
\end{proposition}

\begin{corollary}
$\mathbb{E}[\Psi_A(t)] = ((\mathbb{E}[\Psi_A(0)] + 1)2^{-t} - 1)^+ \leq ((\vert Child(A) \vert + 1)2^{-t} - 1)^+$
\end{corollary}

\begin{proposition}
$\forall B \subseteq A, \;\; \mathbb{E}[\Psi_B(t)] \leq \mathbb{E}[\Psi_A(t)]$
\end{proposition}

\begin{lemma}
$ \forall t > \log_2 \vert Child(A) \vert, \;\; \mathbb{E}[\Psi_{A}(t)] = 0$
\end{lemma}

% CHAPTER 3
\section{Edge Partitions}

Fix a rooted tree $\pazocal{T} = (V,E)$ with root $r$. We define a partition of $E$ that will allow us to tightly bound the total number of right critical insertions for each set in the partition. Let $C_i$ be a subset of $E$ where $(x,y) \in C_i$ if and only if there is a path of length $i$ from $y$ to $r$. It's clear that $C_0, C_1,..., C_D$ partitions $E$ where $D$ is the depth of $\pazocal{T}$. We will show that the expected number of right critical insertions to $r$ in $C_i$ is bounded by $\pazocal{O}(\frac{1}{i}\log{(\vert C_i \vert)})$, which will allow us to bound the expected recourse of $r$ by $\pazocal{O}(\log n\log D)$ and hence the total recourse of $\pazocal{T}$ by $\pazocal{\tilde O}(n)$.

\subsection{Bounding Right Critical Insertions to the root}

Given any edge $e=(x,y)$, we can see that if $e$ is right critical to $r$, then it is also necessarily right critical to all nodes on the path from $y$ to $r$, which means that $y$ can reach $r$. Denote by $A_i$ the set of endpoints of edges in $C_i$, and by $A^*_i \subseteq A_i$ the subset of $A_i$ that can reach $r$ at some given point during the run of $\pazocal{A}$ on $\pazocal{T}$. We see that at any point the root $r$ is to the right of all nodes in $A^*_i$. Using this observation, we bound the expected number of right critical insertions from $C_i$ to $r$ using the following argument.

\[
\mathbb{E} [\textrm{number of right critical insertions from  $C_i$ to $r$}]
\]

\[
= \sum_{e \in C_i} \mathbb{P}[\textrm{$e$ causes a right critical insertion to $r$}]
\]

\[
\leq \sum_{(x,y) \in C_i} \mathbb{P}[\textrm{inserting $(x,y)$ decreases $\Psi_{A_i^*}$ $\land$ $y$ can reach $r$}]
\]

\[
\leq \sum_{(x,y) \in C_i} \mathbb{P}[\textrm{inserting $(x,y)$ decreases $\Psi_{A_i}$ $\land$ $y$ can reach $r$}]
\]

\[
= \sum_{e \in C_i} \mathbb{P}[\textrm{inserting $e$ decreases $\Psi_{A_i}$}] \; \mathbb{P}[\textrm{$e$ causes a critical insertion to $r$}]
\]

\[
= \frac{1}{i} \sum_{e \in C_i} \mathbb{P}[\textrm{inserting $e$ decreases $\Psi_{A_i}$}]
\]

\[
= \frac{1}{i} \mathbb{E}[\textrm{number of insertions from $C_i$ that decrease $\Psi_{A_i}$}]
\]

\[
\leq \frac{1}{i} \log_2(\vert Child(A_i) \vert) = \frac{1}{i} \log_2(\vert C_{i} \vert)
\]

From this we can immediately deduce that the total expected recourse of $r$ is $\pazocal{\tilde O}(n)$.

\[
\mathbb{E} [\textrm{recourse of $r$}] = \sum_{i=1}^D \mathbb{E}[\textrm{number of right critical insertions from $C_i$ to $r$}]
\]

\[
\leq \sum_{i=1}^D \frac{1}{i} \log_2(\vert C_{i} \vert) \leq \sum_{i=1}^D \frac{1}{i} \log_2(n) = H_D\log_2(n) = \pazocal{O}(\log D \log n)
\]

New approach. Fix some $i \in [D]$ and let $\mathds{1}_i(t)$ be the indicator random variable for the event that the $t^{th}$ insertion from $C_i$ causes one of the nodes in 




\end{document}