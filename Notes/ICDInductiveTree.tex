\documentclass{article}

% packages
\usepackage[utf8]{inputenc}
\usepackage{amsthm}
\usepackage{amsmath}
\usepackage{amssymb}
\usepackage{enumerate}
\usepackage{dsfont}

\usepackage{calrsfs}
\DeclareMathAlphabet{\pazocal}{OMS}{zplm}{m}{n}

% definitions and commands
\newtheorem{theorem}{Theorem}
\newtheorem{lemma}{Lemma}
\newtheorem{corollary}{Corollary}
\newtheorem{proposition}{Proposition}
\newtheorem{definition}{Definition}

\newcommand{\norm}[1]{\left\lVert#1\right\rVert}

\voffset = -20pt % defaults to 0pt
\textheight = 581pt % defaults to 561pt
\hoffset = -10pt % defaults to 0pt
\textwidth = 375pt % defaults to 355pt

\title{Expected Recourse Bounds (Draft 1)}
\author{Martin Costa}
\date{August 2020}

\begin{document}

\maketitle

% CHAPTER 1
\section{Introduction}

The main idea of these definitions and propositions will be to give us tools which we can use to analyse the expected recourse of the \textit{simple consistent one-way search} algorithm, which we will denote by $\pazocal{A}$. We assume from now on that $\pazocal{A}$ always performs forward searches, and we fix a DAG $G=(V,E)$.

\begin{definition}
During the run of $\pazocal{A}$ on $G$, we say a node $u$ is \textbf{critical} to $v$ if:
\begin{enumerate}
    \item $u$ cannot reach $v$
    \item there exists some edge $(u,w) \in E$ and $w$ can reach $v$ 
\end{enumerate}
We say an edge $(u,v)$ is \textbf{critical} to $w$ if its insertion increases the amount of nodes that can reach $w$, this implies that $u$ is critical to $w$. Similarly, we say that an edge $(u,v)$ is \textbf{right critical} to $w$ if its insertion causes the recourse of $w$ to increase.
\end{definition}

This definition has the following obvious properties.

\begin{proposition}
Given any $u,v \in V$, if $u$ is \textbf{not} an ancestor of $v$ in $G$, then $u$ can never be critical to $v$ during the run of $\pazocal{A}$ on $G$. 
\end{proposition}

\begin{proposition}
During the run of $\pazocal{A}$ on $G$, the insertion of an edge $(u,v)$ can only increase the recourse of some node $w$ if $(u,v)$ is critical to $w$.
\end{proposition}

Combining these two properties we get the following.

\begin{lemma}
Fix a sequence of edge insertions and an initial ordering for $G$, and use these to induce a sequence of edge insertions and an initial ordering for $H = G[reach^{-1}_{G}(x)]$ for some node $x$. Then given some $u \in reach^{-1}_{G}(x)$, the recourse of $u$ in the run of $\pazocal{A}$ on $G$ equals the recourse of $u$ in the run of $\pazocal{A}$ on $H$.
\end{lemma}

Intuitively, this lemma is saying that the effect of $\pazocal{A}$ on some node $u$ depends only on the structure of the ancestors of $u$. We can immediately notice the following.

\begin{corollary}
In the run of $\pazocal{A}$ on $G$, there always exists some node $u$ whose removal has no effect on the recourse of any other node in the graph.
\end{corollary}

Combining the Lemma and the Corollary it may be possible to approach this analysis inductively, considering only the predecessors of some node $u$ and noticing that their behaviour can not be influenced by the presence of $u$, and then considering the effect that they will have upon $u$. Motivated by this Lemma, I also believe it is the case that the recourse of $u$ is $\pazocal{O}(\log{\vert reach_{G}^{-1}(u) \vert})$, since the removal of any node not in this set from $G$ will not effect the recourse of $u$.

% CHAPTER 2
\section{Memorylessness}

\begin{definition}
Let a \textbf{configuration} of the run of $\pazocal{A}$ on $G$ be a tuple $(S,\prec)$ where $S$ is a subset of $E$ and $\prec$ is a topological ordering of $V$ with respect to $G[S]$. We say a configuration $(S,\prec)$ \textbf{yields} $(S^*,\prec^*)$ if there exists some sequence of edge insertions of $S^* \setminus S$ that can be inserted by $\pazocal{A}$ to obtains order $\prec^*$ starting from $\prec$.
\end{definition}

\begin{definition}
We say an algorithm is \textbf{memoryless} if given any configuration $(S, \prec)$ which yields configurations $(S^*, \prec^*_1)$ and $(S^*, \prec^*_2)$, we have that $\prec^*_1 = \prec^*_2$.
\end{definition}

\begin{proposition}
$\pazocal{A}$ is memoryless on the space of forests of rooted trees.
\end{proposition}

\begin{proof}
We proceed by induction on the depth $D$ of the deepest tree in the forest. The base case $D=0$ is trivial as there are no insertions to be made.

Let $\pazocal{F}$ be a forest of rooted trees with maximum depth $k$ and assume that the proposition holds for all $0 \leq D < k$. Let $X$ be the set of nodes in $\pazocal{F}$ of out degree 0. By Lemma 1, we know that the presence of the $x_i$ has no effect on how $\pazocal{A}$ behaves with the rest of the nodes $Y = V \setminus X$ in $\pazocal{F}$. So, by applying the inductive hypothesis, we get that the positions of any $u,v \in Y$ relative to each other in the ordering is unaffected by the order in which all edges without an endpoint in $X$ are inserted.

Now consider some fixed $x \in X$ with children $y_1,...,y_l$ in $\pazocal{F}$ and consider the following 2 cases.

\begin{enumerate}
    \item $x$ is to the right of all of its ancestors in the initial ordering
    \item $x$ is to the left of one of its ancestors in the initial ordering
\end{enumerate}

In case 1, $x$ is never moved by $\pazocal{A}$, so no insertions with an endpoint in $x$ have any effect on the ordering. In case 2, the position of $x$ in the final ordering is just to the right of the rightmost node out of $y_1,...,y_l$. In either case, the position of $x$ depends only on the edges present in $\pazocal{F}$, never on when they are inserted.
\end{proof}

Using the memoryless property of rooted trees we can prove the following lemma which will be crucial for the next section. First we define a potential function. Fix a rooted tree $\pazocal{T} = (V,E)$ with root $r$, and let $A \subseteq V$ be a set of mutually unrelated nodes in $\pazocal T$, i.e. given any two nodes from $A$, there exists no path between them in $\pazocal T$. Let $\Psi_A \triangleq \vert Child(A) \cap R(A) \vert$, where $Child(A)$ is the set of children of $A$ and $R(A)$ is the set of all nodes to the right of $A$ in the topological ordering. Let $C_A \subseteq E$ denote the set of edges from $Child(A)$ to $A$ and let $\Psi_A(t)$ denote the value of $\Psi_A$ after $t$ edges from $C_A$ have been inserted.

We show that the memoryless property of rooted trees implies the following bounds on the expectation of the potential function. Let $k = \vert C_A \vert$.

\begin{proposition}
\[ \mathbb{E}[\Psi_A(t)] \leq \binom{k}{t}^{-1} \sum_{i=1}^k \binom{i-1}{t-1}(k-1) = \frac{k-t}{t+1} \]
\end{proposition}

\begin{proof}
Suppose that $S \subseteq E$ has been inserted by $\pazocal{A}$ during its run on $\pazocal{T}$ so far, and that $\vert S \cap C_A \vert = t$. We can see that the potential function $\Psi_A$ depends only on the ordering of the nodes. Since, by the memoryless property, the ordering does not depend on the order in which the edges from $S$ are inserted, we deduce that $\Psi_A$ does not depend on this either. So if we can bound $\mathbb{E}[\Psi_A(t)]$ assuming that the edges from $S$ are inserted in some specific order, we get that this bound will hold for all possible insertion sequences of $S$.

Assume that $S$ is inserted such that the final $t$ insertions are those from $C_A$. Then any of the last $t$ edges being inserted can not have any effect on the relative order of the nodes in $Child(A)$. Since the insertions occur uniformly at random, we can show via a counting argument that after the final $t$ insertions from $C_A$, we expect the rightmost node which is incident on one of these edges, say $u$, to have $(k-t)/(t+1)$ nodes from $A$ to its right. Therefore, since some node in $A$ will necessarily be to the right of $u$, $\mathbb{E}[\Psi_A(t)] \leq (k-t)/(t+1)$.
\end{proof}

Say that the insertion of $e$ \textbf{shifts} $A$ if it causes $\Psi_A$ to decrease. Notice that the probability of some edge $e \in C_A$ shifting $A$ is equal to the probability that its startpoint is in $Child(A) \cap R(A)$ at the time of its insertion. Using this we get that

\[  \mathbb{P}[\textrm{the $t^{th}$ insertion from $C_A$ shifts $A$}] = \frac{\mathbb{E}[\Psi_A(t)]}{k-t} \]

Then we get the following result.

\begin{proposition}
\[ \mathbb{E}[\textrm{insertions from $C_A$ that shift $A$}] = \pazocal{O}(\log k) \]
\end{proposition}

\begin{proof}
\[ \mathbb{E}[\textrm{insertions from $C_A$ that shift $A$}] = \sum_{e \in C_A} \mathbb{E}[\textrm{e shifts $A$}] \]

\[ \sum_{e \in C_A} \mathbb{P}[\textrm{e shifts $A$}]  = \sum_{t = 1}^{k} \mathbb{P}[\textrm{the $t^{th}$ insertion from $C_A$ shifts $A$}] \]

\[ \leq \sum_{t = 1}^{k} \frac{\mathbb{E}[\Psi_A(t)]}{k-t} \leq \sum_{t = 1}^{k} \frac{1}{t+1} = H_{k+1} = \pazocal{O}(\log k) \]
\end{proof}

% CHAPTER 3
\section{Edge Partitions}

\subsection{Bounding Right Critical Insertions to the Root}

Fix a rooted tree $\pazocal{T} = (V,E)$ with root $r$. We define a partition of $V$ into sets of mutually unrelated nodes, which we can use to tightly bound the total number of right critical insertions to $r$ for each set in the partition. Let $A_i$ be a subset of $V$ where $x \in A_i$ if and only if there is a path of length $i-1$ from $x$ to $r$. It is clear that $A_1,..., A_{D+1}$ partitions $V$ into sets of mutually unrelated nodes where $D$ is the depth of $\pazocal{T}$. Letting $C_i$ denote all edges with endpoints in $A_i$, we can see that $C_1,...,C_D$ forms a partition of $E$.

We will now prove the following

\[
\mathbb{E} [\textrm{right critical insertions from  $C_i$ to $r$}]
\]

\[
= \sum_{e \in C_i} \mathbb{P}[\textrm{$e$ causes a right critical insertion to $r$}]
\]

\[
\leq \sum_{e \in C_i} \mathbb{P}[\textrm{e shifts $A_i$ $\land$ $e$ can reach $r$}] = \sum_{e \in C_i} \mathbb{P}[\textrm{e shifts $A_i$}] \mathbb{P}[\textrm{$e$ can reach $r$}]
\]

\[
= \sum_{e \in C_i} \mathbb{P}[\textrm{e shifts $A_i$}] \mathbb{P}[\textrm{$e$ is critical to $r$}] = \sum_{e \in C_i} \mathbb{P}[\textrm{e shifts $A_i$}] \frac{1}{i}
\]

\[
=\mathbb{E}[\textrm{insertions from $C_i$ that shift $A_i$}] \frac{1}{i} = \pazocal{O}(\log \vert C_i \vert)\frac{1}{i}
\]

\end{document}