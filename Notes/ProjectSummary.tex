\documentclass{article}

% packages
\usepackage[utf8]{inputenc}
%\usepackage[linesnumbered,lined,boxed,commentsnumbered]{algorithm2e}
\usepackage[ruled,vlined,linesnumbered]{algorithm2e}
\usepackage{amsthm}
\usepackage{amsmath}
\usepackage{amssymb}
\usepackage{enumerate}
\usepackage{dsfont}

\usepackage{calrsfs}
\DeclareMathAlphabet{\pazocal}{OMS}{zplm}{m}{n}

% definitions and commands
\newtheorem{theorem}{Theorem}[section]
\newtheorem{lemma}[theorem]{Lemma}
\newtheorem{conjecture}[theorem]{Conjecture}
\newtheorem{corollary}[theorem]{Corollary}
\newtheorem{proposition}[theorem]{Proposition}
\newtheorem{definition}[theorem]{Definition}

\newcommand{\norm}[1]{\left\lVert#1\right\rVert}
\newcommand{\polylog}{\: \mathrm{polylog} \:}

\voffset = -20pt % defaults to 0pt
\textheight = 581pt % defaults to 561pt
\hoffset = -10pt % defaults to 0pt
\textwidth = 375pt % defaults to 355pt

\title{CS344 Discrete Mathematics Project \\  - Incremental Cycle Detection - }
\author{Martin Costa}
\date{August 2020}

\begin{document}

\maketitle

\tableofcontents

\section{Introduction}

Give an overview of the problem and what I plan on covering in this text.

In this project I will be discussing variants of two different problems on the topic of incremental cycle detection, which I shall refer to as the \textbf{data structure} and \textbf{recourse} problems. I will start by giving formal descriptions of the problems, and discuss some of the important results on the topic of incremental cycle detection and it's connection to incremental topological ordering. I will then survey some state of the art algorithms for the data structure problem, and then discuss the recourse problem as well as giving some original results.

\section{Problem Definitions}

\subsection{Topological Order Maintenance}

Show that incremental cycle detection algorithms rely on topological ordering maintenance algorithms.

\subsection{Data Structure Problem}

Give formal descriptions of the data structure problem, and define relevant notation.

\subsection{Recourse Problem}

Give formal descriptions of the recourse problem, and define relevant notation.

\subsection{Simple Algorithms}

Define local algorithms and describe the simple search algorithms such as the one-way and two-way greedy search algorithms.

\section{Recourse Variant and My Contributions}

\subsection{Conjectures}

Go over the conjectures about the recourse problem such as polylog update time under random order arrival for simple algorithms.

Up to now we have been considering the problem of incremental cycle detection with \textit{adversarial insertions}, where we must handle \textit{any} arbitrary insertion sequence. We shall now consider the problem in the \textit{random order arrival} model, where the insertion sequence is generated uniformly at random.

With random order insertions, it is the believed that following conjecture is true.

\begin{conjecture}
Let $G=(V,E)$ be a DAG. The expected total recourse of a simple algorithm on $G$ is $\pazocal O(n \: \mathrm{ polylog } \: n)$, with expectation taken over insertion sequences.
\end{conjecture}

\subsection{Simple Recourse Upper Bound}

Give lower bound on recourse for local algorithms under adversarial arrival. Show that this instance is tight.

\subsection{Random Order Arrival}

Show that this lower bound does not hold under random order arrival.

\subsection{Trees Under Adversarial Arrival}

Show that one-way search has low recourse under adversarial arrival.

While doing research on this problem, I was able to prove the following special case under adversarial insertions.

\begin{theorem}
Let $\pazocal T = (V,E)$ be a tree. The expected total recourse of the simple one-way search algorithm on $\pazocal T$ is $\pazocal O(n \log n)$, with expectation taken over the initial ordering.
\end{theorem}

\section{References}

[HKMST11] "Incremental Cycle Detection, Topological Ordering, and Strong Component Maintenance" https://arxiv.org/pdf/1105.2397.pdf

\end{document}