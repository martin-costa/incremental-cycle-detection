\documentclass{article}

% packages
\usepackage[utf8]{inputenc}
\usepackage{amsthm}
\usepackage{amsmath}
\usepackage{amssymb}
\usepackage{enumerate}
\usepackage{dsfont}

\usepackage{calrsfs}
\DeclareMathAlphabet{\pazocal}{OMS}{zplm}{m}{n}

% definitions and commands
\newtheorem{theorem}{Theorem}
\newtheorem{lemma}{Lemma}
\newtheorem{corollary}{Corollary}
\newtheorem{proposition}{Proposition}
\newtheorem{definition}{Definition}

\newcommand{\norm}[1]{\left\lVert#1\right\rVert}

\voffset = -20pt % defaults to 0pt
\textheight = 581pt % defaults to 561pt
\hoffset = -10pt % defaults to 0pt
\textwidth = 375pt % defaults to 355pt

\title{Expected Recourse Bounds (Draft 1)}
\author{Martin Costa}
\date{August 2020}

\begin{document}

\maketitle

% CHAPTER 1
\section{Introduction}

These definitions and propositions will be to give us tools which we can use to analyse the expected recourse of the \textit{simple consistent one-way search} algorithm, which we will denote by $\pazocal{A}$. We assume from now on that $\pazocal{A}$ always performs forward searches, and we fix a DAG $G=(V,E)$.

\begin{definition}
During the run of $\pazocal{A}$ on $G$, we say a node $u$ is \textbf{critical} to $v$ if:
\begin{enumerate}
    \item $u$ cannot reach $v$
    \item there exists some edge $(u,w) \in E$ and $w$ can reach $v$ 
\end{enumerate}
We say an edge $(u,v)$ is \textbf{critical} to $w$ if its insertion increases the amount of nodes that can reach $w$, this implies that $u$ is critical to $w$. Similarly, we say that an edge $(u,v)$ is \textbf{right critical} to $w$ if its insertion causes the recourse of $w$ to increase.
\end{definition}

This definition has the following obvious properties.

\begin{proposition}
Given any $u,v \in V$, if $u$ is \textbf{not} an ancestor of $v$ in $G$, then $u$ can never be critical to $v$ during the run of $\pazocal{A}$ on $G$. 
\end{proposition}

\begin{proposition}
During the run of $\pazocal{A}$ on $G$, the insertion of an edge $(u,v)$ can only increase the recourse of some node $w$ if $(u,v)$ is critical to $w$.
\end{proposition}

Combining these two properties we get the following.

\begin{lemma}
Fix a sequence of edge insertions and an initial ordering for $G$, and use these to induce a sequence of edge insertions and an initial ordering for $H = G[reach^{-1}_{G}(x)]$ for some node $x$. Then given some $u \in reach^{-1}_{G}(x)$, the recourse of $u$ in the run of $\pazocal{A}$ on $G$ equals the recourse of $u$ in the run of $\pazocal{A}$ on $H$.
\end{lemma}

Intuitively, this lemma is saying that the effect of $\pazocal{A}$ on some node $u$ depends only on the structure of the ancestors of $u$. We can immediately notice the following.

\begin{corollary}
In the run of $\pazocal{A}$ on $G$, there is always at least one node whose removal has no effect on the recourse of any other node in the graph and there is always at least one node which must have a recourse of 0.
\end{corollary}

Combining the Lemma and the Corollary we can assume without loss of generality that there exists some root node $r$ such that $G = G[reach^{-1}(r)]$. Showing that the expected recourse of $r$ is $\pazocal{O}(\textrm{polylog} \: n)$ would give us that the expected recourse of $\pazocal A$ on any DAG is $\pazocal{\tilde O}(n)$. 

% CHAPTER 2
\section{The Idea}

Fix a DAG $G=(V,E)$ with root $r$.

\begin{definition}
Let $(e_t)_{t \in [m]}$ be any insertion sequence of $G$. We say that the insertion sequence $(e_t)_{t \in [m]}$ is \textbf{normal} if for all $x \in V$ we have that at least one edge from $out_G(x)$ is inserted before any edges from $in_G(x)$, unless $out_G(x) = \varnothing$.
\end{definition}

\begin{corollary}
Let $(e_t)_{t \in [m]}$ be any normal insertion sequence of $G$. The edge $(x,y)$ is inserted after $y$ can already reach $r$.
\end{corollary}

\begin{corollary}
Let $(e_t)_{t \in [m]}$ be any normal insertion sequence of $G$. Given any $x \in V$ such that $r \prec_t x$, we have that $out_t(x) = \varnothing$.
\end{corollary}

This corollary says that any node $x$ appearing to the right of $r$ in the ordering $\prec_t$ cannot have any outgoing edges in $G_t$.

\begin{corollary}
Let $(e_t)_{t \in [m]}$ be any normal insertion sequence of $G$. For any $0 \leq t < t^* \leq m$ and $x \in V$, if $x \prec_t r$ then $x \prec_{t^*} r$.
\end{corollary}

This corollary says nodes can not be moved from the left of $r$ to the right of $r$ if the insertions sequence is normal.

It should be clear that these normal insertion sequences have many desirable properties, making it easy to prove the following Lemma.

\begin{lemma}
Let $(e_t)_{t \in [m]}$ be any normal insertion sequence of $G$. The expected recourse of $r$ is $\pazocal{O}(\log n)$, with expectation taken over the uniformly generated initial ordering.
\end{lemma}

The following Lemma is at the core of this approach. Let $rec_x((e_t)_{t \in [m]}, \prec_0)$ denote the recourse of $x$ in the run of $\pazocal A$ on $G$ when inserting $(e_t)_{t \in [m]}$ starting with initial ordering $\prec_0$.

\begin{lemma}
Let $(e_t)_{t \in [m]}$ be any insertion sequence of $G$ and fix any initial ordering $\prec_0$. We can compute a normal insertion sequence $\Gamma((e_t)_{t \in [m]}) = (\hat{e}_t)_{t \in [m]}$ such that we have $rec_r((e_t)_{t \in [m]}, \prec_0) \leq rec_r((\hat{e}_t)_{t \in [m]}, \prec_0)$.
\end{lemma}

Finally, combining the last 2 Lemmas gives us the following...

\begin{theorem}
Let $(e_t)_{t \in [m]}$ be any insertion sequence of $G$, then

\[ \mathbb{E}[ rec_r((e_t)_{t \in [m]})] \leq \mathbb{E}[ rec_r(\Gamma((e_t)_{t \in [m]}))] = \pazocal{O}(\log n) \]
\end{theorem}

Using the ideas from the last chapter, we can bound the total recourse by $\pazocal{O}(n \log n)$ by applying the argument above to each node in the graph individually.
\end{document}